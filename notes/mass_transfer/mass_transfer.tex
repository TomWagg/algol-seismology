\documentclass[12pt]{article}
\usepackage[letterpaper, margin=0.8in]{geometry}
\usepackage{nopageno,epsfig, amsmath, amssymb}
\usepackage{physics}
\usepackage{mathtools}
\usepackage{hyperref}
\usepackage{xcolor}
\hypersetup{
    colorlinks,
    linkcolor={blue},
    citecolor={blue},
    urlcolor={blue}
}
\usepackage{empheq}
\usepackage{wrapfig}

\usepackage{fancyhdr, titling}

\usepackage{parskip}
\setlength{\parindent}{0pt}

\title{Mass Transfer Rates}
\author{\textbf{Tom Wagg}}

\pagestyle{fancy}
\fancyhf{}
\rhead{\theauthor}
\lhead{\thetitle}
\rfoot{Page \thepage}

\newcommand{\tomtitle}{
    \noindent {\LARGE \fontfamily{cmr}\selectfont \textbf{\thetitle}} \hfill \\[1\baselineskip]
    \noindent {\large \fontfamily{cmr}\selectfont Kavli Summer Program \hfill \textsc{Tom Wagg}}\\[0.5\baselineskip]
}

\newcommand{\question}[1]{{\noindent \it #1}}
\newcommand{\answer}[1]{
    \par\noindent\rule{\textwidth}{0.4pt}#1\vspace{0.5cm}
}
\newcommand{\todo}[1]{{\color{red}\begin{center}TODO: #1\end{center}}}

% custom function for adding units
\makeatletter
\newcommand{\unit}[1]{%
    \,\mathrm{#1}\checknextarg}
\newcommand{\checknextarg}{\@ifnextchar\bgroup{\gobblenextarg}{}}
\newcommand{\gobblenextarg}[1]{\,\mathrm{#1}\@ifnextchar\bgroup{\gobblenextarg}{}}
\makeatother

\newcommand{\avg}[1]{\left\langle #1 \right\rangle}
\newcommand{\angstrom}{\mbox{\normalfont\AA}}
\allowdisplaybreaks

\begin{document}

\tomtitle{}

\thispagestyle{empty}

Let's consider what the mass transfer rates for our stars should be. First let's just write some of the basic timescales. The nuclear timescale we can reduce assuming H core fusion and the mass-luminosity relation.
\begin{align}
    \tau_{\rm nuc} &\sim \frac{M f_M c^2 (\Delta m/m)}{L}\\
    &\sim 7.1 \times 10^{-4} \frac{M c^2}{L}\\
    &\sim 7.1 \times 10^{-4} \frac{\unit{M_{\odot}} c^2}{1.4 \unit{L_{\odot}}} \qty(\frac{M}{\unit{M_{\odot}}})^{-2.5} \\
    \Aboxed{ \tau_{\rm nuc} &\sim 7.5 \times 10^{9} \unit{yr} \qty(\frac{M}{\unit{M_{\odot}}})^{-2.5} }
\end{align}
Then the thermal timescale is approximately:
\begin{align}
    \tau_{\rm therm} &\sim \frac{G M^2}{R L} \\
    &\sim \frac{G \unit{M_{\odot}^2}}{1.4 \unit{R_\odot} \unit{L_\odot}} \qty(\frac{R}{\unit{R_\odot}})^{-1} \qty(\frac{M}{\unit{M_\odot}})^{-1.5} \\
    \Aboxed{ \tau_{\rm therm} &\sim 2.2 \times 10^{7} \unit{yr} \qty(\frac{R}{\unit{R_\odot}})^{-1} \qty(\frac{M}{\unit{M_\odot}})^{-1.5} }
\end{align}

Now we can consider some different limits. Assuming that we accrete on a timescale of $\tau_{\rm accrete}$ (and that $\tau_{\rm therm} \ll \tau_{\rm nuc}$):
\begin{itemize}
    \item $\tau_{\rm accrete} \gg \tau_{\rm nuc}$: If the accretion is much slower than the nuclear timescale then the star will exhaust a significant fraction (or all!) of its central hydrogen before accretion can complete. This is not very realistic.
    \item $\tau_{\rm accrete} \sim \tau_{\rm nuc}$:
    \item $\tau_{\rm accrete} \sim \tau_{\rm therm}$:
    \item $\tau_{\rm accrete} \ll \tau_{\rm therm}$:
\end{itemize}

\end{document}

 