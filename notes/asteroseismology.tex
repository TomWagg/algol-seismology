\documentclass[12pt]{article}
\usepackage[letterpaper, margin=0.8in]{geometry}
\usepackage{nopageno,epsfig, amsmath, amssymb}
\usepackage{physics}
\usepackage{mathtools}
\usepackage{hyperref}
\usepackage{xcolor}
\hypersetup{
    colorlinks,
    linkcolor={blue},
    citecolor={blue},
    urlcolor={blue}
}
\usepackage{empheq}
\usepackage{wrapfig}

\usepackage{fancyhdr, titling}

\usepackage{parskip}
\setlength{\parindent}{0pt}

\title{Asteroseismology quick notes}
\author{\textbf{Tom Wagg}}

\pagestyle{fancy}
\fancyhf{}
\rhead{\theauthor}
\lhead{\thetitle}
\rfoot{Page \thepage}

\newcommand{\class}{Asteroseismology Quick Notes}

\newcommand{\tomtitle}{
    \noindent {\LARGE \fontfamily{cmr}\selectfont \textbf{\class}} \hfill \\[1\baselineskip]
    \noindent {\large \fontfamily{cmr}\selectfont Kavli Summer Program \hfill \textsc{Tom Wagg}}\\[0.5\baselineskip]
}

\newcommand{\question}[1]{{\noindent \it #1}}
\newcommand{\answer}[1]{
    \par\noindent\rule{\textwidth}{0.4pt}#1\vspace{0.5cm}
}
\newcommand{\todo}[1]{{\color{red}\begin{center}TODO: #1\end{center}}}

% custom function for adding units
\makeatletter
\newcommand{\unit}[1]{%
    \,\mathrm{#1}\checknextarg}
\newcommand{\checknextarg}{\@ifnextchar\bgroup{\gobblenextarg}{}}
\newcommand{\gobblenextarg}[1]{\,\mathrm{#1}\@ifnextchar\bgroup{\gobblenextarg}{}}
\makeatother

\newcommand{\avg}[1]{\left\langle #1 \right\rangle}
\newcommand{\angstrom}{\mbox{\normalfont\AA}}
\allowdisplaybreaks

\begin{document}

\tomtitle{}

\thispagestyle{empty}

\textbf{Brunt Vaisala Frequency:} Determines the buoyancy.
\begin{equation}
    N^2 = g \qty(\frac{1}{\Gamma_1 P} \dv{P}{r} - \frac{1}{\rho} \dv{\rho}{r})
\end{equation}
Or for a fully-ionised ideal gas:
\begin{equation}
    N^2 \approxeq \frac{g^2 \rho}{P} \qty(\grad_{\rm ad} - \grad + \grad_\mu)
\end{equation}
$N^2 > 0$ means you get oscillations about the equilibrium, otherwise you get convective instabilities.

\textbf{Convective regions:} Gravity waves \textit{cannot} propagate in convective regions. Recall that $M > 1.2 \unit{M_{\odot}}$ stars have convective cores (and this covers the entire mass range for this project). Larger stars will have larger convective cores.

\textbf{Lamb Frequency:} This seems to also be referred to as the characteristic acoustic frequency.
\begin{equation}
    S_{l}^2 = \frac{l(l + 1) c_s^2}{r^2}
\end{equation}

\textbf{$p$ modes and $g$ modes:} $p$ modes have high frequencies above both $N$ and $S_l$, whilst $g$ modes have low frequencies below both $N$ and $S_l$. Any intervening regions have waves exponentially increasing/decreasing as a function of $r$.

\end{document}

 