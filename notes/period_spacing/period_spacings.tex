\documentclass[12pt]{article}
\usepackage[letterpaper, margin=0.8in]{geometry}
\usepackage{nopageno,epsfig, amsmath, amssymb}
\usepackage{physics}
\usepackage{mathtools}
\usepackage{hyperref}
\usepackage{xcolor}
\usepackage{natbib}
\hypersetup{
    colorlinks,
    linkcolor={blue},
    citecolor={blue},
    urlcolor={blue}
}
\usepackage{empheq}
\usepackage{wrapfig}

\usepackage{fancyhdr, titling}

\usepackage{parskip}
\setlength{\parindent}{0pt}

\usepackage{../aas_macros}

\title{Physics of Period Spacing Patterns}
\author{\textbf{Tom Wagg}}

\pagestyle{fancy}
\fancyhf{}
\rhead{\theauthor}
\lhead{\thetitle}
\rfoot{Page \thepage}

\newcommand{\tomtitle}{
    \noindent {\LARGE \fontfamily{cmr}\selectfont \textbf{\thetitle}} \hfill \\[1\baselineskip]
    \noindent {\large \fontfamily{cmr}\selectfont Kavli Summer Program \hfill \textsc{Tom Wagg}}\\[0.5\baselineskip]
}

\newcommand{\question}[1]{{\noindent \it #1}}
\newcommand{\answer}[1]{
    \par\noindent\rule{\textwidth}{0.4pt}#1\vspace{0.5cm}
}
\newcommand{\todo}[1]{{\color{red}\begin{center}TODO: #1\end{center}}}

% custom function for adding units
\makeatletter
\newcommand{\unit}[1]{%
    \,\mathrm{#1}\checknextarg}
\newcommand{\checknextarg}{\@ifnextchar\bgroup{\gobblenextarg}{}}
\newcommand{\gobblenextarg}[1]{\,\mathrm{#1}\@ifnextchar\bgroup{\gobblenextarg}{}}
\makeatother

\newcommand{\avg}[1]{\left\langle #1 \right\rangle}
\newcommand{\angstrom}{\mbox{\normalfont\AA}}
\allowdisplaybreaks

\begin{document}

\tomtitle{}

\thispagestyle{empty}

Here are some notes to ensure I actually understand what I'm talking about :D

Let's start with an equation, because that's how my brain works. I think this is called the asymptotic approximation. Anyway under the assumption of spherical symmetry (we're ignoring rotation folks) and high-order: the g-mode period spacing (i.e. the difference in period between modes of the same spherical degree and neighbouring radial order) is approximately constant and given by \citep[e.g.,][]{Hatta+2023}
\begin{equation}
    \Delta P_g = \frac{\pi^2}{\sqrt{l(l+1)}} \qty[\int_{r_0}^{r_1} \frac{N}{r} \dd{r}]^{-1},
\end{equation}
where $l$ is the spherical degree and
\begin{equation}
    N^2 \approxeq \frac{g^2 \rho}{P} \qty(\grad_{\rm ad} - \grad + \grad_\mu),
\end{equation}
is the Brunt Vaisala (BV) frequency which is integrated within the g-mode cavity. Recall that $\grad_{\rm ad}$ is a constant (equal to 2/5 for an ideal gas), $\grad$ is the temperature gradient and $\grad_\mu$ is the chemical composition gradient.

This means that the period spacing, for a fixed spherical degree, is dependent on the location of
\begin{itemize}
    \item[(i)] the core boundary (I'm assuming the cavity extends to the edge of the star so $r_1$ is fixed)
    \item[(ii)] the BV frequency profile
\end{itemize}
For more massive stars, the size of the convective core increases, which reduces the size of the g-mode cavity and thus decreases the integral. This tends to dominate over any changes in the BV profile and so the overall period spacing increases almost monotonically along with stellar evolution \citep{Miglio+2008}.

In addition to these effects, we haven't even thought about rotation yet! This is a rather glaring omission since we expect some pretty fast rotation for our target stars it seems. Under the traditional approximation of rotation (TAR), the period spacing is quasi-linearly related to period rather than constant, and the rate of this rotation determines the gradient of the relation \citep{Bouabid+2013}. So essentially instead of a flat line we expect increasingly slanted lines for faster rotating stars. I \textit{think} that Cole also mentioned that the gradient of this line will flip for retrograde rotators.

\bibliographystyle{aasjournal}
\bibliography{refs}

\end{document}

 